\documentclass[10pt,a4paper]{article}
\usepackage[utf8]{inputenc}
\usepackage[english]{babel}
\usepackage[T1]{fontenc}
\usepackage{amsmath}
\usepackage{amsfonts}
\usepackage{amssymb}
\usepackage{makeidx}
\usepackage{graphicx}
\usepackage{fourier}
\usepackage{listings}
\usepackage{color}
\usepackage{hyperref}
\usepackage[left=2cm,right=2cm,top=2cm,bottom=2cm]{geometry}
\author{Johannes Scheller, Vincent Noculak, Lukas Powalla}
\title{Computational Physics - Project 3}
\begin{document}
\lstset{language=C++,
	keywordstyle=\bfseries\color{blue},
	commentstyle=\itshape\color{red},
	stringstyle=\color{green},
	identifierstyle=\bfseries,
	frame=single}
\maketitle
\newpage
\tableofcontents
\newpage
\section{Introduction to Project 3}

	
\section{Introduction to Project 3}

We are looking at the six-dimensional integral, which is used to determine the ground state correlation energy between to electrons in a helium atom. This integral is given by: 
\begin{equation}
	I = \int\limits_{\mathbb{R}^6} d\mathbf{r_1} d\mathbf{r_2} e^{-4 (r_1+r_2)} \frac{1}{|\mathbf{r_1} - \mathbf{r_2}|} 
\end{equation}

Or in spherical coordinates:

\begin{align}
	\begin{split}
		I = \int_0^\infty \int_0^\infty  \int_0^{2 \pi} \int_0^{2 \pi}  \int_0^\pi \int_0^\pi dr_1 dr_2 d\theta_1 d\theta_2 d\phi_1 d\phi_2    r_1^2 r_2^2 sin(\theta_1) sin(\theta_2)\\ \cdot \frac{e^{-4(r_1+r_2)}} {\sqrt{r_1^2+r_2^2-r_1r_2(cos(\theta_1)cos(\theta_2)+sin(\theta_1)sin(\theta_2)cos(\phi_1-\phi_2))}}
	\end{split}
\end{align}

The analytical solution of this integral is $I = \frac{5 \pi^2}{16^2}$.

To solve this integral numerical. We will first apply the Gauss-Legendre quadrature for every variable in Cartesian coordinates. After that we use the Gauss-Laguerre quadrature in Spherical coordinates. Next we will study the solution for the Monte Carlo method. Where we first apply a brute force algorithm in Cartesian coordinates and then improve the algorithm with importance sampling, by eliminating the exponential term of the integral.

\subsection{Analytical derivation of the integral }

We want to calculate the analytical solution to the integral:
\begin{align*}
I = \int d\vec{r}_1 \int d\vec{r}_2 \frac{1}{|\vec{r_1}-\vec{r_2}|} \cdot e^{-4(r_1+r_2)}
\end{align*}
Therefore, we first transform to spherical Coordinates for each variable. In addition to that, we choose while calculating the integral over $r_2$ the axis of $r_1$ as z-axis. With simple scalar product ( $\vec{r}_1 \cdot \vec{r}_2= r_1 \cdot r_2 \cdot cos( \theta)$) we get the following expression: (don't forget about Jacobi-determinant)
\begin{align}
I&=\int_{0}^{2 \pi} d\phi_1 \int_{0}^{ \pi} sin(\theta_1) d\theta_1 \int_{0}^{\infty}  r_1^2 \cdot e^{-4 \cdot r_1} dr_1\cdot \int_{0}^{2 \pi} d\phi_2 \int_{0}^{ \pi} sin(\theta_2) d\theta_2 \int_{0}^{\infty}  r_2^2 dr_2 \frac{1}{\sqrt{r_1^2+r_2^2-2 \cdot r_1 \cdot r_2\cdot cos(\theta_2)}} \cdot e^{-4 \cdot r_2}\\
&= 4 \pi^2 \cdot \int_{0}^{ \pi} sin(\theta_1) d\theta_1 \int_{0}^{\infty}  r_1^2 \cdot e^{-4 \cdot r_1} dr_1 \cdot \quad \int_{0}^{\infty} dr_2 \int_{0}^{\pi} d\theta_2 \cdot \frac{r_2^2 \cdot sin(\theta_2)}{\sqrt{r_1^2+r_2^2-2 \cdot r_1 r_2 cos(\theta_2)}}\cdot e^{-4 \cdot r_2}\\
&= 4 \pi^2 \cdot \int_{0}^{ \pi} sin(\theta_1) d\theta_1 \int_{0}^{\infty}  r_1^2 \cdot e^{-4 \cdot r_1} dr_1 \cdot \quad I(\vec{r}_1)_2
\end{align}
We now calculate the integral $I(\vec{r}_1)_2$:
\begin{align}
I(\vec{r}_1)_2&= \int_{0}^{\infty} dr_2 \int_{0}^{\pi} d\theta_2 \cdot \frac{r_2^2 \cdot sin(\theta_2)\cdot e^{-4 \cdot r_2}}{\sqrt{r_1^2+r_2^2-2 \cdot r_1 r_2 cos(\theta_2)}}\\
&= \int_{0}^{\infty} dr_2 \cdot r_2^2\cdot e^{-4 \cdot r_2} \cdot \left[ \frac{1}{r_1 r_2} \sqrt{r_1^2+r_2^2 - 2 r_1 r_2 cos(\theta_2)} \right]_0^{\pi}\\
&= \int_{0}^{\infty} dr_2 \cdot r_2^2 \left( \sqrt{r_1^2+r_2^2+2 \cdot r_1 \cdot r_2} -\sqrt{r_1^2+r_2^2-2 \cdot r_1 \cdot r_2}\right) \cdot e^{-4r_2}\\ 
&= \int_{0}^{\infty} dr_2 \frac{r_2}{r_1} \left( r_1+r_2 - |r_1-r_2| \right) \cdot e^{-4 \cdot r_2}
\end{align}
Now, we split up the integral in two parts:
\begin{align}
I(\vec{r}_1)_2&=\int_{0}^{r_1} dr_2 2 \cdot \frac{r_2^2}{r_1} e^{-4 r_2} + 2 \cdot\int_{r_1}^{\infty} dr_2 e^{-4r_2}\\
&= \frac{2}{r_1} \cdot \hat{I}_1 +  \hat{I}_2
\end{align}
Through partial integration, we can calculate the following integral:
\begin{align}
\hat{I}_1&=\int_{0}^{r_1} dr_2 \cdot r_2^2 e^{-4 r_2}\\
&= -\frac{1}{4} r_1^2 e^{-4r_2}|_0^{r_1} +\int_0^{r_1} \frac{1}{2} r_2 e^{-4 r_2}dr_2\\
&=-\frac{1}{4}r_1^2 e^{-4 r_1}-\frac{1}{8}r_2 e^{-4r_2} |_0^{r_1} + \int_0^{r_1} \frac{1}{8} e^{-4 r_2} dr_2\\
&=-\frac{1}{4}r_1^2 e^{-4 r_1}-\frac{1}{8}r_1 e^{-4r_1} + \frac{1}{32}-\frac{1}{32} \cdot e^{-4 r_1}\\
\end{align}
Similarly, we determine the second integral:
\begin{align}
\hat{I}_2 &= 2 \cdot\int_{r_1}^{\infty} dr_2 e^{-4r_2}\\
&= -\frac{1}{2} r_2 e^{-4 r_2} |_{r_1}^{\infty} + \int_{r_1}^{\infty} \frac{1}{2} e^{-4 r_2} dr_2 \\
&=\frac{1}{2} r_1 e^{-4 r_1} + \frac{1}{8} e^{-4 r_1} 
\end{align}
In total, we can now determine the integral $I_2$:
\begin{align}
I_2 &=  \frac{2}{r_1}\left(-\frac{1}{4}r_1^2 e^{-4 r_1}-\frac{1}{8}r_1 e^{-4r_1} + \frac{1}{32}-\frac{1}{32} \cdot e^{-4 r_1}\right) + \frac{1}{2} r_1 e^{-4 r_1} + \frac{1}{8} e^{-4 r_1} \\
&= -\frac{1}{16}\,{\frac { \left( 2\,{{\rm e}^{-
4\,r_1}}r_1+{{\rm e}^{-4\,r_1}}-1 \right) }{r_1}}
\end{align}
Finally, we can calculate the integral:
\begin{align}
I&= 4 \pi^2 \cdot \int_{0}^{ \pi} sin(\theta_1) d\theta_1 \int_{0}^{\infty}  r_1^2 \cdot e^{-4 \cdot r_1} dr_1 \left( -\frac{1}{16}\,{\frac { \left( 2\,{{\rm e}^{-
4\,r_1}}r_1+{{\rm e}^{-4\,r_1}}-1 \right) }{r_1}} \right)\\
&= 8 \pi^2 \cdot \int_{0}^{\infty}  r_1^2 \cdot e^{-4 \cdot r_1} dr_1 \left( -\frac{1}{16}\,{\frac { \left( 2\,{{\rm e}^{-
4\,r_1}}r_1+{{\rm e}^{-4\,r_1}}-1 \right) }{r_1}}\right)\\
&=-\frac{1}{2}\pi^2 \cdot \int_0^{\infty} e^{-4 \cdot r_1} dr_1 \left( 2 e^{-4r_1} \cdot r_1^2 + e^{-4r_1} \cdot r_1 - r_1 \right)\\
&=-\frac{1}{2}\pi^2 \cdot \int_0^{\infty}  dr_1 \left( 2 e^{-8r_1} \cdot r_1^2 + e^{-8r_1} \cdot r_1 - r_1 \cdot e^{-4 \cdot r_1} \right)\\
\end{align}
We can derive with partial integration the expression:
\begin{align}
\int_{0}^{\infty} dx \cdot x^n \cdot e^{- \beta x } = \frac{n!}{\beta^{n+1}}
\end{align}
Then, we can calculate the integral as follows:
\begin{align}
I&= -\frac{1}{2} \pi^2 \cdot \left[ \frac{2 \cdot 2!}{8^{2+1}}+ \frac{1!}{8^{1+1}} - \frac{1!}{4^{1+1}} \right]\\
&= -\frac{1}{2} \pi^2 \cdot \left[ \frac{4}{8 \cdot8 \cdot 8}+ \frac{8}{8 \cdot 8 \cdot 8} - \frac{1}{8 \cdot 2} \right]\\
&= -\frac{1}{2} \pi^2 \cdot \left[ \frac{12}{8 \cdot8 \cdot 8} - \frac{32}{8 \cdot 8 \cdot 8} \right]\\
&= \pi^2 \cdot \left[ \frac{16}{8 \cdot 8 \cdot 8}- \frac{6}{8 \cdot 8 \cdot 8} \right]= \pi^2 \frac{10}{8 \cdot 8 \cdot 8}= \pi^2 \frac{5}{16^2}\approx 0.19277
\end{align}
We have now derived an analytical expression for the integral. The integral has the value:
\begin{align}
I = \frac{5 \pi^2}{16^2}
\end{align}

\section{Theoretical background for numerical integration}

\subsection{Gaussian quadrature}

\subsection{Montecarlo integration}

\section{Execution}

\subsection{Gaussian quadrature}

\subsection{Montecarlo integration}

We calculated the same integral with montecarlo method. First, we calculated the integral in a brute force way. This means that we calculate the integral using (pseudo) random numbers, which obey uniform distribution functions. The random numbers for each of the six dimensional integral are uniform distributed in a chosen interval (-a to a ). Furthermore, we don't transform the integral, but we calculate it in Cartesian coordinates. We got the values in table \ref{Data from the brute force montecarlo algorithm (part c))}. (We used the interval for a=2 ) In one dimension, the montecarlo method can be described by formula \ref{integralbrute}. 

\begin{align}
I &= \int_{a}^{b} f(x) dx \approx < f(x) > \cdot (b-a) = \frac{1}{n} \sum_{i=1}^{n} f(x_i) \cdot (b-a) = \hat{I} \label{integralbrute}
\end{align}
 
In addition to that, we tried to improve our calculations. First, we transformed the integral into spherical coordinates. In spherical coordinates, we use the variables $\theta$ ( from 0 to $\pi$), $\phi$ (from 0 to 2$\pi$) and r (from 0 to infinity) instead of using Cartesian coordinates $x_{i,k} \ [-\infty$ t $\infty$ ) (i=1,2,3; k=1,2). We also used a distribution function in order to get appropriate values of the random numbers.  Formula \ref{P(x)} to \ref{with distribution} describe the general one dimensional reformulation if you want to use a other particle distribution function. 


\begin{align}
&P(x) = \int_{0}^{x} p(x) dx \label{P(x)}\\
& I =  \int_{a}^{b} \frac{f(x)}{p(x)} \cdot p(x) dx = \cdot \int_{a}^{b} \hat{f}(x) \cdot p(x) dx \approx  \frac{1}{n} \sum_{i=1}^{n} \hat{f}(y_i) \cdot (b-a) = \hat{I} \\
&y_i(x_i) =P^{-1}\left(p(y(x_i))\right)= P^{-1}(x_i)\label{with distribution}
\end{align}
In order to improve the precision of the integral, we used a not uniform distribution function, which can be found in formula \ref{inourcase}ff. 
\begin{align}
P(x)&= \int_{0}^{x} 4 \cdot e^{-4 x} = 1- e^{-4x} \label{inourcase}\\
y_i(x_i)&= -\frac{1}{4} ln(1-x_i)
\end{align}
We transformate the integral to spherical Coordinates and use the distribution function for $r_1$ and $r_2$. Finally, the integral can be calculated through formula \ref{integralinourcase}. The results are in table \ref{Data from the  montecarlo algorithm (part d))}.
\begin{align}
f( r_{1,i}, r_{2,i}, \theta_{1,i}, \theta_{2,i}, \phi_{1,i}, \phi_{2,i}) = \frac{r_{1,i}^2 \cdot r_{2,i}^2 \cdot sin(\theta_{1,i}) sin(\theta_{2,i}) }{\sqrt{r_{1,i}^2+r_{2,i}^2-2 \cdot r_{1,i} r_{2,i}  cos(\theta_{1,i}) cos(\theta_{2,i}) + sin(\theta_{1,i}) sin(\theta_{2,i}) \cdot cos(\phi_{1,i}-\phi_{2,i})}\cdot 4^2}
\end{align}

\begin{align}
\hat{I}= \frac{1}{n} \sum_{i=1}^{n} f( r_{1,i}, r_{2,i}, \theta_{1,i}, \theta_{2,i}, \phi_{1,i}, \phi_{2,i}) \cdot (2 \pi - 0)^2 \cdot (\pi -0)^2  \label{integralinourcase}
\end{align}

\begin{table}[h]
\centering
\caption{Data from the brute force montecarlo algorithm (part c))}
\label{Data from the brute force montecarlo algorithm (part c))}
\begin{tabular}{c|c|c|c}
n & Integral & standart deviation & time in s \\
\hline\hline
100 & 0.0563094 & 0.0362644 & 0 \\
1000 & 0.226437 & 0.159513 & 0.001 \\
10000 & 0.0986332 & 0.0263534 & 0.006 \\
100000 & 0.155652 & 0.01757 & 0.062 \\
1000000 & 0.178452 & 0.00754909 & 0.667 \\
10000000 & 0.188683 & 0.00290854 & 6.645 \\
100000000 & 0.19169 & 0.000942637 & 70.881 
\end{tabular}
\end{table}

\begin{table}[h]
\centering
\caption{Data from the montecarlo algorithm with distribution function (in spherical coordinates) (part d))}
\label{Data from the  montecarlo algorithm (part d))}
\begin{tabular}{c|c|c|c}
n & Integral & standart deviation & time in s \\
\hline\hline
100 & 0.216734 & 0.0769193 & 0 \\
1000 & 0.173694 & 0.0241284 & 0.002 \\
10000 & 0.186069 & 0.00885492 & 0.021 \\
100000 & 0.19571 & 0.00343622 & 0.21 \\
1000000 & 0.193259 & 0.00101078 & 2.07 
\end{tabular}
\end{table}

\section{Comparison and discussion of the results}

\section{Source-code}


\end{document}