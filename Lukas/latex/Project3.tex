\documentclass[10pt,a4paper]{article}
\usepackage[utf8]{inputenc}
\usepackage[english]{babel}
\usepackage[T1]{fontenc}
\usepackage{amsmath}
\usepackage{amsfonts}
\usepackage{amssymb}
\usepackage{makeidx}
\usepackage{graphicx}
\usepackage{fourier}
\usepackage{listings}
\usepackage{color}
\usepackage{hyperref}
\usepackage[left=2cm,right=2cm,top=2cm,bottom=2cm]{geometry}
\author{Johannes Scheller, Vincent Noculak, Lukas Powalla}
\title{Computational Physics - Project 3}
\begin{document}
\lstset{language=C++,
	keywordstyle=\bfseries\color{blue},
	commentstyle=\itshape\color{red},
	stringstyle=\color{green},
	identifierstyle=\bfseries,
	frame=single}
\maketitle
\newpage
\tableofcontents
\newpage
\section{Introduction to Project 3}

\section{Theoretical background for numerical integration}

\subsection{Gaussian quadrature}

\subsection{Montecarlo integration}

\section{Execution}

\subsection{Gaussian quadrature}

\subsection{Montecarlo integration}

We calculated the same integral with montecarlo method. First, we calculated the integral in a brute force way. This means that we calculate the integral using (pseudo) random numbers, which obey uniform distribution functions. The random numbers for each of the six dimensional integral are uniform distributed in a chosen interval (-a to a ). Furthermore, we don't transform the integral, but we calculate it in Cartesian coordinates. We got the values in table \ref{Data from the brute force montecarlo algorithm (part c))}. (We used the interval for a=2 ) In one dimension, the montecarlo method can be described by formula \ref{integralbrute}. 

\begin{align}
I &= \int_{a}^{b} f(x) dx \approx < f(x) > \cdot (b-a) = \frac{1}{n} \sum_{i=1}^{n} f(x_i) \cdot (b-a) = \hat{I} \label{integralbrute}
\end{align}
 
In addition to that, we tried to improve our calculations. First, we transformed the integral into spherical coordinates. In spherical coordinates, we use the variables $\theta$ ( from 0 to $\pi$), $\phi$ (from 0 to 2$\pi$) and r (from 0 to infinity) instead of using Cartesian coordinates $x_{i,k} \ [-\infty$ t $\infty$ ) (i=1,2,3; k=1,2). We also used a distribution function in order to get appropriate values of the random numbers.  Formula \ref{P(x)} to \ref{with distribution} describe the general one dimensional reformulation if you want to use a other particle distribution function. 


\begin{align}
&P(x) = \int_{0}^{x} p(x) dx \label{P(x)}\\
& I =  \int_{a}^{b} \frac{f(x)}{p(x)} \cdot p(x) dx = \cdot \int_{a}^{b} \hat{f}(x) \cdot p(x) dx \approx  \frac{1}{n} \sum_{i=1}^{n} \hat{f}(y_i) \cdot (b-a) = \hat{I} \\
&y_i(x_i) =P^{-1}\left(p(y(x_i))\right)= P^{-1}(x_i)\label{with distribution}
\end{align}
In order to improve the precision of the integral, we used a not uniform distribution function, which can be found in formula \ref{inourcase}ff. 
\begin{align}
P(x)&= \int_{0}^{x} 4 \cdot e^{-4 x} = 1- e^{-4x} \label{inourcase}\\
y_i(x_i)&= -\frac{1}{4} ln(1-x_i)
\end{align}
We transformate the integral to spherical Coordinates and use the distribution function for $r_1$ and $r_2$. Finally, the integral can be calculated through formula \ref{integralinourcase}. The results are in table \ref{Data from the  montecarlo algorithm (part d))}.
\begin{align}
f( r_{1,i}, r_{2,i}, \theta_{1,i}, \theta_{2,i}, \phi_{1,i}, \phi_{2,i}) = \frac{r_{1,i}^2 \cdot r_{2,i}^2 \cdot sin(\theta_{1,i}) sin(\theta_{2,i}) }{\sqrt{r_{1,i}^2+r_{2,i}^2-2 \cdot r_{1,i} r_{2,i}  cos(\theta_{1,i}) cos(\theta_{2,i}) + sin(\theta_{1,i}) sin(\theta_{2,i}) \cdot cos(\phi_{1,i}-\phi_{2,i})}\cdot 4^2}
\end{align}

\begin{align}
\hat{I}= \frac{1}{n} \sum_{i=1}^{n} f( r_{1,i}, r_{2,i}, \theta_{1,i}, \theta_{2,i}, \phi_{1,i}, \phi_{2,i}) \cdot (2 \pi - 0)^2 \cdot (\pi -0)^2  \label{integralinourcase}
\end{align}

\begin{table}[h]
\centering
\caption{Data from the brute force montecarlo algorithm (part c))}
\label{Data from the brute force montecarlo algorithm (part c))}
\begin{tabular}{c|c|c|c}
n & Integral & standart deviation & time in s \\
\hline\hline
100 & 0.0563094 & 0.0362644 & 0 \\
1000 & 0.226437 & 0.159513 & 0.001 \\
10000 & 0.0986332 & 0.0263534 & 0.006 \\
100000 & 0.155652 & 0.01757 & 0.062 \\
1000000 & 0.178452 & 0.00754909 & 0.667 \\
10000000 & 0.188683 & 0.00290854 & 6.645 \\
100000000 & 0.19169 & 0.000942637 & 70.881 
\end{tabular}
\end{table}

\begin{table}[h]
\centering
\caption{Data from the montecarlo algorithm with distribution function (in spherical coordinates) (part d))}
\label{Data from the  montecarlo algorithm (part d))}
\begin{tabular}{c|c|c|c}
n & Integral & standart deviation & time in s \\
\hline\hline
100 & 0.216734 & 0.0769193 & 0 \\
1000 & 0.173694 & 0.0241284 & 0.002 \\
10000 & 0.186069 & 0.00885492 & 0.021 \\
100000 & 0.19571 & 0.00343622 & 0.21 \\
1000000 & 0.193259 & 0.00101078 & 2.07 
\end{tabular}
\end{table}

\section{Comparison and discussion of the results}

\section{Source-code}


\end{document}