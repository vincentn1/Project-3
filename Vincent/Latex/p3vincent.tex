\documentclass[10pt,a4paper]{article}

\usepackage[utf8]{inputenc}
\usepackage[english]{babel}
\usepackage[T1]{fontenc}
\usepackage{amsmath}
\usepackage{amsfonts}
\usepackage{amssymb}
\usepackage{makeidx}
\usepackage{graphicx}
\usepackage{fourier}
\usepackage{listings}
\usepackage{color}
\usepackage[left=2cm,right=2cm,top=2cm,bottom=2cm]{geometry}
\usepackage{hyperref}

\author{Johannes Scheller, Vincent Noculak, Lukas Powalla}
\title{Computational Physics - Project 3}


\begin{document}

\maketitle
\newpage
\tableofcontents
\newpage

\begin{abstract}
	
In this project, we look at different numerical integration methods applied on the problem of the correlation energy of two electrons in a helium atom. First we will study the Gauss-Legendre and Gauss-Laguerre quadrature and compare them to each other. Later we will evaluate the integral using a Monte Carlo calculation, which we will improve by using importance sampling. The usefulness of the different methods will be discussed.
	
\end{abstract}

\section{Introduction}

We are looking at the six-dimensional integral, which is used to determine the ground state correlation energy between to electrons in a helium atom. This integral is given by: 
\begin{equation}
I = \int\limits_{\mathbb{R}^6} d\mathbf{r_1} d\mathbf{r_2} e^{-4 (r_1+r_2)} \frac{1}{|\mathbf{r_1} - \mathbf{r_2}|} 
\end{equation}

Or in spherical coordinates:

\begin{align}
	\begin{split}
    I = \int_0^\infty \int_0^\infty  \int_0^{2 \pi} \int_0^{2 \pi}  \int_0^\pi \int_0^\pi dr_1 dr_2 d\theta_1 d\theta_2 d\phi_1 d\phi_2    r_1^2 r_2^2 sin(\theta_1) sin(\theta_2)\\ \cdot \frac{e^{-4(r_1+r_2)}} {\sqrt{r_1^2+r_2^2-r_1r_2(cos(\theta_1)cos(\theta_2)+sin(\theta_1)sin(\theta_2)cos(\phi_1-\phi_2)}}
   	\end{split}
\end{align}

The analytical solution of this integral is $I = \frac{5 \pi^2}{16^2}$.

To solve this integral numerical. We will first apply the Gauss-Legendre quadrature for every variable in Cartesian coordinates. After that we use the Gauss-Laguerre quadrature in Spherical coordinates. Next we will study the solution for the Monte Carlo method. Where we first apply a brute force algorithm in Cartesian coordinates and then improve the algorithm with importance sampling, by eliminating the exponential term of the integral.

\section{Theory}
\subsection{Gauss-Legendre and Gauss-Laguerre quadrature}

In Gaussian quadrature we use the characteristics of orthogonal polynomials to numerically integrate a function. The Legendre polynomials which are of such kind are defined in the interval $[-1;1]$. Using mapping we can use the Gauss-Legendre quadrature for an integral with any integration limits $a, b \in \mathbb{R}$. In our case we can make the approximation $\int_{\infty}^{\infty} f(\mathbf{r_1},\mathbf{r_2}) \approx \int_{-a}^{a} f(\mathbf{r_1},\mathbf{r_2})$ in case "a" is high enough, because the value of our function decreases very quickly due to the exponential function. The value of the numerically calculated integral in six dimensions is given by:

\begin{align}
I = \sum_{f,g,h,i,j,k = 1}^{n}\omega_f \cdot \omega_g \cdot \omega_h \cdot \omega_i \cdot \omega_j \cdot \omega_k \cdot f(x_{1,f} , x_{2,g} , y_{1,h} , y_{2,i} , z_{1,j} , z_{2,k})  
\end{align}

Where $f$ is the function we want to integrate, $x_i$ is a point where the Legendre polynomial of degree n is zero and $\omega_i$ can be seen a the weight of this point.

Gauss-Laguerre quadrature is especially good to numerically solve integrals of the form $\int_{0}^{\infty} e^{-\alpha x} f(x) dx$. As seen in (2), we have such an integral if we use Spherical coordinates. Hence we will integrate $r_1$ and $r_2$ in Spherical coordinates with Gauss-Laguerre, while we integrate $\theta_1$, $\theta_2$, $\phi_1$ and $\phi_2$ with Gauss-Legendre quadrature. The formula will have the same form as (3).

\subsection{Monte-Carlo-Method} 

In the Monte-Carlo integration we make use of the fact, that you can write an integral of a function as the expectation value of this function with the uniform distribution.

\begin{equation}
I = \int_{a}^{b} f(x) dx = <f> \approx \frac{(b-a)}{n} \sum_{i=1}^{n} f(x_i)
\end{equation}

Hence in our case we can write our integral as:

\begin{equation}
	I = \frac{(2a)^6}{n} \sum_{i = 1}^n f(x_{1,i}, x_{2,i}, y_{1,i}, y_{2,i}, z_{1,i}, z_{2,i})
\end{equation}

Where the $x_i$'s, $y_i$'s and $z_i$'s are random generated numbers in the interval $[-a;a]$ ($a \in \mathbb{R}$) with an uniform distribution. If we choose "a" big enough this is a good approximation to the integration limits(which are infinite), because the value of our function decreases quickly. 
If we use importance sampling in the Monte Carlo method in Spherical coordinates, we can eliminate the exponential term in our function and also do not need to integrate with the infinity as a limit any more. We make an importance sampling with the functions

\begin{align}
\rho(r_1) = e^{-4 r_1} \\\rho(r_2) = e^{-4 r_2}
\end{align}

As a consequence our integral will transform to 
\begin{align}
\begin{split}
	    I = \int_0^\frac{1}{4} \int_0^\frac{1}{4}  \int_0^{2 \pi} \int_0^{2 \pi}  \int_0^\pi \int_0^\pi dr_1 dr_2 d\theta_1 d\theta_2 d\phi_1 d\phi_2    ln(1-4 r_1)^2 ln(1-4 r_2)^2 sin(\theta_1) sin(\theta_2)\\ \cdot \frac{1} {\sqrt{ln(1-4 r_1)^2+ln(1-4 r_2)^2-ln(1-4 r_1)ln(1-4 r_2)(cos(\theta_1)cos(\theta_2)+sin(\theta_1)sin(\theta_2)cos(\phi_1-\phi_2)}}
\end{split}
\end{align}

The integral gets calculated in the same way, we did it before the importance sampling.







\end{document}